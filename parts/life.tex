\section{Rezepte}
\subsection{Fondue}
Fondue à la Doeme:

\subsubsection{Zutaten}
Für eine Person:
\begin{itemize}
\item 125g Vacherin
\item 125g Greyerzer
\item Weisswein
\item Kirschwasser
\item Maisstärke
\item Knoblauch
\end{itemize}
\subsubsection{Rezept}

\begin{enumerate}
\item Fonduecaquelon~\footnote{Eine Pfanne} mit ein paar Knoblauchzehen ausreiben.
\item Caquelon mit dem Knoblauch etwas erhitzen, und Weisswein zugeben (ca. 3mm $\frac {\mbox{Füllhöhe}} {\mbox{Person}}$ bei normaler Caquelondimension [$\sim$22cm])~\footnote{Etwas Weisswein}.
\item Nach weiterem kurzen Erwärmen den geriebenen Käse zugeben.
\item Die Mischung mit Rührbewegungen der Form einer 8 folgend zum globalen Schmelzpunkt begleiten~\footnote{Rühren bis die Brühe flüssig ist.}.
\item Sobald das Fondue flüssig ist, Maisstärke in einem Glässchen Kirschgeist zu einer Suspension verrühren, und selbige anschliessend dem Fondue beifügen.
\item Den Fonduecaquelon geographisch auf den Rechaud replazieren.
\item \emph{Solange das Fondue erwärmt wird, muss es permanent mit einem Löffel oder Löffelersatz~\footnote{Gabel mit Brotaufsatz} gerührt werden.} 
\end{enumerate}

\subsection{Met-Hollunder-Schorle}
\begin{itemize}
\item $2^{0}$ Anteile Hollundersirup
\item $2^{2}$ Anteile Met
\item $2^{3}$ Anteile Mineralwasser, spritzig
\end{itemize}
\subsection{Gute heisse Schokolade}
\label{sec:hot_chocolate}
Mengen müssen selber evaluiert werden!
\begin{itemize}
	\item Milch.~\footnote{\url{http://www.youtube.com/watch?v=w4aLThuU008}}
	\item Kakaopulver(Kein Schokoladenpulver, das reine).
	\item Mascobadozucker, bzw. brauner Zucker.
	\item Schokoladenpulver.
\end{itemize}
Das Verhältnis von $\frac{\mbox{Schokoladenpulver}}{\mbox{Kakaopulver}+\mbox{Mascobadozucker}}$ sollte unter $\frac{1}{1}$ liegen, muss aber wie bereits erwähnt selbst ermittelt werden.
\subsection{Chocolat Rum}
\begin{itemize}
	\item Schokolade gemäss Kapitel~\ref{sec:hot_chocolate}, Seite~\pageref{sec:hot_chocolate}.
	\item Einen Schuss Rum.
	\item Ein wortwörtliches Sahnehäubchen.
\end{itemize}
\section{*.Körper}
\subsection{Haare}
\subsubsection{Kokosöl}
Wenn die Haare sehr trocken sind/neigen sich zu verknoten, kann es helfen, die Längen vor dem Duschen mit Kokosöl (echtes) einzufetten. Alternativ ins trockene Haar eine sehr kleine Menge einarbeiten.